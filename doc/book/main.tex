%% awa5x - Extended AWA5.0
%% Copyright © 2024 Starknights

%% This program is free software: you can redistribute it and/or modify
%% it under the terms of the GNU General Public License as published by
%% the Free Software Foundation, either version 3 of the License, or
%% (at your option) any later version.

%% This program is distributed in the hope that it will be useful,
%% but WITHOUT ANY WARRANTY; without even the implied warranty of
%% MERCHANTABILITY or FITNESS FOR A PARTICULAR PURPOSE.  See the
%% GNU General Public License for more details.

%% You should have received a copy of the GNU General Public License
%% along with this program.  If not, see <https://www.gnu.org/licenses/>.
\documentclass[11pt,a4paper,draft]{book}
\usepackage[T1]{fontenc}
\usepackage[utf8]{inputenc}
\usepackage[autostyle]{csquotes}
\usepackage{verbatim}
\usepackage{parskip}
\usepackage{polyglossia}
\usepackage[colorlinks,hyperindex]{hyperref}
\setdefaultlanguage{english}

\author{The Starcnight Initiatibe}
\title{The Extended AWA5 Programming Language}

\begin{document}
\frontmatter
\tableofcontents

\chapter{Preface}
(to be written.)

\mainmatter
\chapter{Introduction to Awatism}
An old adage says:
\textquote{The only way to learn a new programming language is by writing programs in it.};
thus, we'll begin our exploration of the language by presenting some
practical programs with the goal of letting you, the reader, reach the
point where you can write useful programs as quickly as possible.

To reach this goal, we will avoid getting lost in tedious details or
boring rules; rather, we will present the basics of correct programs
through working examples all the while providing only the necessary
informations to understand the given code.

On the other hand, omitting certain details and keeping the examples
simple can lead to misleading results: precaution ought to be taken
when issuing certain commands to the machine and possible techniques
to create good code might not be presented at all. The authors
encourage the reader to reach past what is shown in this chapter---or
even in this book as a whole---when starting a new project,
experiment, and find ways to improve.

\section{Greetings}
It is customary to send our regards to the world when learning a new
programming language; as such, we will now write a program which will
print these words:
\begin{verbatim}
Hello, world!
\end{verbatim}

It is assumed a basic writing environment is available to the reader,
such as being able to use a plain text editor, and that the tools
shown later are installed and available on the reader's machine.

In AWA5, the program to print \verb|Hello, world!| is:
\verbatiminput{hello1.awa}

That's an overwhelming listing of seemingly nonsensical words. For
now, let's try to get this program running: as a specific example
using the \emph{awa5x} suite of tools, create a file named
\verb|hello.awa|, then compile it as follows:
\begin{verbatim}
awa5c hello.awa > hello.wa
\end{verbatim}

If all went well, the compilation process will silently generate a
file called \verb|hello.wa|. This file is our \enquote{Hello world}
program and it can be run as follows:
\begin{verbatim}
awa5 hello.wa
\end{verbatim}

which will print:
\begin{verbatim}
Hello, world!
\end{verbatim}

Let's now examine the code. Every AWA5 program is written by composing
together two syllables: \emph{awa} and \emph{wa}; the first syllable
is to be written in isolation, following any amount of blanks, the
second one must immediately follow a syllable with no blanks
inbetween.

To better understand this rule, let's examine the second line of the
code listing:
\begin{verbatim*}
awa awawa awawa awa awawawa awawawawa
\end{verbatim*}

Each blank, indicated by the \verb*| | symbol,
separate each \emph{awa} syllable from the others;
the \emph{wa} syllable is instead connected to the
previous \emph{awa} or \emph{wa} syllable.

To be recognized as an AWA5 program, the code must begin with a single
\emph{awa} syllable. In our \enquote{Hello world} program it is the
very first line. A list of \emph{statements} follow this opening word,
each specifying an operation to be performed by the
machine. Statements are executed in order from top to bottom and the
program will terminate once the last command is reached.

Every AWA5 program operates on a hidden \emph{stack} of values:
whenever something is \emph{pushed} on the stack it will be considered
the \emph{top} value; when programs \emph{pop} from the stack, this
top value is removed. Pushing a value on the stack will not make the
previous value disappear; rather, it will be stored \enquote{below}
the top value until a pop command is executed. The stored value will
then become the new top value.

Our \enquote{Hello world} program contains both a \emph{push} and a
\emph{pop} command, repeated multiple times in order to generate the
desired effect: the first few lines will push each letter of the
\verb|Hello, world!| sentence on the stack, then each value is printed
and automatically popped off.

It is important to highlighting that each character is inserted in
reversed order: the print command operates on the last inserted value,
thus to write our greetings correctly the \verb|H| letter must be
inserted last.

\textbf{Excercise 1-1}. Run the \enquote{Hello world} program. Try to
omit characters or lines and see what happens each time you compile
it. \\
\textbf{Excercise 1-2}. Change the order of each line after the first
(remember, \emph{awa} must be the very first word) and see the
result. Try adding more \emph{pop} command than \emph{push} ones and
try to answer why it doesn't work.
\end{document}
