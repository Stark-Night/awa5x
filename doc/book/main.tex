%% awa5x - Extended AWA5.0
%% Copyright © 2024 Starknights

%% This program is free software: you can redistribute it and/or modify
%% it under the terms of the GNU General Public License as published by
%% the Free Software Foundation, either version 3 of the License, or
%% (at your option) any later version.

%% This program is distributed in the hope that it will be useful,
%% but WITHOUT ANY WARRANTY; without even the implied warranty of
%% MERCHANTABILITY or FITNESS FOR A PARTICULAR PURPOSE.  See the
%% GNU General Public License for more details.

%% You should have received a copy of the GNU General Public License
%% along with this program.  If not, see <https://www.gnu.org/licenses/>.
\documentclass[12pt,a4paper,draft]{book}
\usepackage[T1]{fontenc}
\usepackage[utf8]{inputenc}
\usepackage{polyglossia}
\usepackage[hidelinks]{hyperref}
\usepackage[autostyle]{csquotes}
\usepackage{verbatim}
\setdefaultlanguage{english}

\author{The Starcnight Initiatibe}
\title{The Extended AWA5 Programming Language}

\begin{document}
\frontmatter
\tableofcontents

\chapter{Preface}
(to be written.)

\mainmatter
\chapter{Introduction to Awatism}
An old adage says:
\textquote{The only way to learn a new programming language is by writing programs in it.};
thus, we'll begin our exploration of the language by presenting some
practical programs with the goal of letting you, the reader, reach the
point where you can write useful programs as quickly as possible.

To reach this goal, we will avoid getting lost in tedious details or
boring rules; rather, we will present the basics of correct programs
through working examples all the while providing only the necessary
informations to understand the given code.

On the other hand, omitting certain details and keeping the examples
simple can lead to misleading results: precaution ought to be taken
when issuing certain commands to the machine and possible techniques
to create good code might not be presented at all. The authors
encourage the reader to reach past what is shown in this chapter---or
even in this book as a whole---when starting a new project,
experiment, and find ways to improve.

\section{Greetings}
It is customary to send our regards to the world when learning a new
programming language; as such, we will now write a program which will
print these words:
\begin{verbatim}
Hello, world!
\end{verbatim}

It is assumed a basic writing environment is available to the reader,
such as being able to use a plain text editor, and that the tools
shown later are installed and available on the reader's machine.

In AWA5, the program to print \verb|Hello, world!| is:
\verbatiminput{hello1.awa}
\end{document}
